\section{Introduction} \label{sec-introduction}
% Opening
Machine learning (ML) plays an important role in business, industry, and academia. 
Developing effective ML applications requires knowledge in statistics, big data, and ML systems as well as domain expertise.
Therefore, ML application development is rarely an individual effort and requires collaborations between different ML users.
To this end, recent efforts attempt to enable easy collaboration between ML users.
Platforms such as AzureML \cite{team2016azureml}, Kaggle \cite{kagglewebsite}, and Google Colabratory \cite{googlecolab} provide a collaborative environment where users share their scripts and results in Jupyter notebooks \cite{Kluyver:2016aa}.
Other platforms such as OpenML \cite{vanschoren2014openml} and ModelDB \cite{vartak2016m} enable collaboration by storing machine pipelines, hyperparameters, models, and evaluation results in databases, commonly referred to as experiment databases \cite{Vanschoren2012}.

% P
The collaborative platforms act as repositories of ML artifacts and execution platforms for ML workloads, i.e., scripts or pipelines.
ML artifacts refer to raw or intermediate datasets or ML models.
By automatically exploiting the stored ML artifacts, the collaborative platforms can improve the performance of future workloads by skipping redundant operations and by warmstarting the model training operations using existing models.
However, the existing collaborative platforms ignore the stored artifacts and require the users to manually search through the stored ML artifacts and incorporate them into their workloads.
In the current collaborative environments, we identify two challenges which prohibit the platforms from automatically utilizing the existing ML artifacts.
First, the quantity and size of the generated artifacts are large, which renders their storage not feasible.
Second, naively searching through all the stored artifacts for reuse incurs a substantial overhead.


% S
We propose a solution which addresses these two challenges.
Our solution only stores the ML artifacts with a high likelihood of reappearing in the future workloads.
Furthermore, our solution organizes the ML artifacts and offers a fast reuse procedure.

We model an ML workload with a directed acyclic graph (DAG), where vertices represent the ML artifacts and edges represent the operations in the workload.
Each ML artifact is uniquely identified using the sequence of operations that generated it.
An ML artifact comprises of two components: meta-data and underlying content.
Meta-data refers to the column names of a dataframe, hyperparameters of a model, and evaluation score of a model on a testing dataset.
Underlying content refers to the actual data inside a dataframe or the weight vector of a machine learning model.
We refer to the union of all the workload DAGs as the \textit{Experiment Graph}.
The Experiment Graph is itself a DAG which contains all the vertices (ML artifacts) and edges (operations) of the past workload graphs.
The size of the artifact meta-data is small.
Thus, Experiment Graph stores the meta-data of all the artifacts.
However, there are two scenarios where storing the underlying data in the Experiment Graph is not suitable, i.e., storage capacity is limited and when recreating an artifact is faster than storage/retrieval of the artifact.
We propose an algorithm for materializing the underlying data of the artifacts given a storage budget.
Our materialization algorithm utilizes several metrics such as the size, access frequency, operation run-time, and the score of the machine learning models to decide what artifacts to store.

To optimize the execution of future workloads, we propose efficient artifact reuse and model warmstarting methods.
The process of reuse and warmstarting is as follows.
Given a workload DAG, before executing the operations to generate an artifact, we search for the artifact in the Experiment Graph.
If the Experiment Graph contains the artifact, we reuse the artifact instead of executing the operations.
When the artifact in question is an ML model, due to the stochasticity of some model training operations, we cannot always reuse an existing model from the Experiment Graph.
Instead, we warmstart the training operation of the workload with a model artifact from the Experiment Graph.
Two factors impact the efficiency of the reuse and warmstarting methods, i.e., the number of the artifacts in the Experiment Graph and the storage method of the Experiment Graph (in memory, on disk, or in a remote location).
Therefore, a brute-force artifact search may create a large overhead.
In our reuse and warmstarting methods, we propose early stopping and pruning techniques as well as heuristics to speed up the search and decrease the optimization overhead.

In summary, we make the following contributions:
\begin{itemize}
\item We propose a system for optimizing the execution of machine learning workloads in collaborative environments.
\item We present Experiment Graph, a graph representation of the artifacts and operations of the machine learning workloads.
\item We propose algorithms for materializing the artifacts in the Experiment Graph under limited storage capacity.
\item We propose efficient reuse and warmstarting methods for new workloads.
\end{itemize}

The rest of this paper is organized as follows.
In Section \ref{sec-background}, we provide some background information and show an example.
We introduce our proposed collaborative workload optimizer in Section \ref{sec-ml-workloads}.
In Sections \ref{sec-materialization} and \ref{sec-reuse-and-warmstarting}, we introduce the artifacts materialization algorithms, reuse strategy, and the warmstarting technique. 
In Section \ref{sec-evaluation}, we show the result of our evaluations.
In Section \ref{sec-background}, we discuss the related work and finally, we conclude this work in Section \ref{sec-conclusion}.