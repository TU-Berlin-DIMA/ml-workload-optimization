\section{Related Work} \label{sec-related-work}
\begin{itemize}
\item repository of ml experiments
%Experiment databases include data and meta-data of different data analytics and machine learning experiments executed over time \cite{miao2018provdb, vanschoren2014openml, schelter2017automatically, vartak2016m}.
%They include different information about datasets, data processing pipeline components, machine learning models, execution of machine learning training algorithms, and quality of the models. 
%Moreover, some experiment databases allow users to store some of the artifacts generated during the execution of a workload, such as raw datasets, intermediate datasets (resulting from applying data transformation operations), and machine learning models and their hyperparameters.
%However, due to limited storage space, experiment databases cannot store every artifact.
% 
%Experiment databases can help in designing a better future workload.
%For example, users can query the database to find the answer to the following questions: what type of data transformations and model training operations are executed on a dataset and what is the accuracy of the final models?
%As a result, users can avoid executing data transformations or model training operations that do not result in high-quality models.
%Moreover, experiment databases enable reproducibility and validation of results.
%For example, users can query information about the environment and list of operations in a specific workload.
%As a result, users can re-execute the workload and compare the results.
\item Scientific workflow systems?
\item Hyperparameter tuning, random and grid search
\item Dataset versioning and materialization \cite{bhattacherjee2015principles, vartak2018mistique}
\item Transfer learning
\end{itemize}


