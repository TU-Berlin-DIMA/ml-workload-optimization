\section{Conclusions} \label{sec-conclusion}
We present a system for optimizing machine learning workloads in collaborative environments.
We propose a graph representation of the artifacts of ML workloads, which we refer to as the Experiment Graph.
Using EG, we offer materialization and reuse algorithms.
We propose two materialization algorithms.
The heuristics-based algorithm stores artifacts of the graph based on their likelihood of reappearing in future workloads.
The storage-aware algorithm takes deduplication information of the artifacts into account when materializing them.
Given the set of materialized artifacts, for a new workload, our reuse algorithm finds the optimal execution plan in linear time.

We show that our collaborative optimizer improves the execution time of ML workloads by more than one order of magnitude for repeated executions and by 50\% for modified workloads.
We also show that our storage-aware materialization can store up to 8 times more artifacts than the heuristics-based materialization algorithm.
Our reuse algorithm finds the optimal execution plan in linear-time and outperforms the state-of-the-art polynomial-time reuse algorithm.

\textbf{Future work.}
EG contains valuable information about the meta-data and hyperparameters of the feature engineering and model training operations.
In future work, we plan to utilize this information to automatically construct ML pipelines and tune hyperparameters \cite{Feurer15, thornton2013auto, shang2019democratizing}; thus, fully or partially automating the process of designing ML pipelines.

\begin{acks}
This work was funded by the German Ministry for Education and Research as BIFOLD - Berlin Institute for the Foundations of Learning and Data (ref. 01IS18025A and ref. 01IS18037A) and German Federal Ministry for Economic Affairs and Energy, Project ”ExDRa” (01MD19002B).
\end{acks}