\section{Materialization and Reuse of Operations}\label{sec-materializaiton-and-reuse}
With the experiment graph constructed, we can look for optimization opportunities for feature engineering and model building operations.
The experiment graph contains all the historical data operations.
By materializing the result of the historical operations we can reuse them for future workloads.
We propose three optimizations, namely, \textit{reuse}, \textit{warmstarting}, \textbf{partial-warmstarting}.

\subsection{Materialization of the Experiment Graph}
Since every historical operation may not be useful, we need to devise a mechanism for selecting the most useful operation results (nodes in the graph) to materialize.
Since we do not know the implementation details of the user-defined operations, we do not materialize nodes resulting from the user-defined feature engineering and aggregation operations.
Every edge (operation) in the experiment graph has a label which specifies its execution time in seconds, its size in byte, the number of times it is executed.
Furthermore, some paths in the graph may end with a model.
For every edge in such a path, we also include information about the quality metrics of the model at the end of the path.
Using these four metrics (i.e., execution time, size, frequency, and quality) we devise a cost model that decides which intermediate datasets and machine models should be materialized and which ones can be recreated by executing the operation.
\todo[inline]{Come up with a cost model, something similar to MISTIQUE \cite{vartak2018mistique}}

\subsection{Reuse Optimization for Feature Engineering Operations and Model building}
Having our materialized experiment graph, we now devise a simple strategy to detect reuse opportunities for the feature engineering operations and model training operations of the new machine learning workloads.
Every edge can be uniquely identified in terms of the input columns and the operation.
Before executing a workload, we first utilize the same strategy for constructing the experiment graph, to transform the workload at hand into the graph representation.
Using the graph representation, we then search in the materialized graph for edges that have the same input columns (starting vertex) and operation type.
If the end-vertex of such edges are materialized, we can directly access them and skip executing the operation.
Both feature processing and model building operations can benefit from this optimization.

\subsubsection{Non-deterministic operations}
Some feature engineering operations are non-deterministic (such as count vectorizer).
Therefore, before the workload is executed we cannot know the resulting columns of such operations.
However, as described earlier, edges can be uniquely identified by the input vertex and the operation.
Therefore, for non-deterministic operations, we can look whether the same edge exists in the experiment graph and infer the output vertex without explicitly executing the operation.

%\todo[inline]{Is this even a problem?}
%\subsubsection{Random seed for model building operations}
%Many of model building operations (such as machine learning training operations) rely on random initialization of model parameters or random shuffling of the training data.
%By reusing the existing model building operations, we are discarding the random behavior.
%In some cases, random initialization of the model parameters may yield in a sub-optimal machine learning model which is only converged to a local optima.
%If all the subsequent workloads reuse this existing model, they all have sub-optimal machine learning models.
%We alleviate this problem using two approaches.

\subsection{Warmstarting Optimization For Model Training Operations}
Model training operations include extra hyperparameters that must be selected before the training procedure begins.
Two training operations on the same set of data (graph node) using the same training algorithm could potentially have very different results based on the selected hyperparameters.
Therefore, we cannot apply the reuse optimization in cases where the hyperparameters are different.
Instead, we apply the \textit{warmstarting} optimization.
In the warmstarting optimization, we set the model parameters of the given workload to the model parameters from the experiment database.
Warmstarting can greatly reduce the total running time and increase the quality of the final machine learning model [citation].
However, the stopping criteria play an important role in determining the effect of the warmstarting for the current workload.
For iterative training algorithms that are minimizing a loss function, there are three stopping criteria that specify when to stop the training, namely, the convergence tolerance, the number of iterations, or a combination of the two.

\subsubsection{Convergence tolerance termination criteria}
When the termination criteria is a convergence tolerance value, two scenarios may occur:
\begin{itemize}
\item 1) An existing model in the experiment database has already reached the convergence tolerance
\item 2) No model in the experiment database has reached the convergence tolerance
\end{itemize}
In the first scenario, we expect a large improvement in the training time. 
Unless the hyperparameter setting does not cause the training algorithm to diverge, the training will stop after one iteration.
In the second scenario, We also expect an improvement in the training time.
Warmstarting the model to the parameters of the model with the highest quality will ensure that the training operation will converge using fewer iterations.

\subsubsection{Augmenting the experiment graph}
Once the training procedure is finished, we augment the experiment graph with an edge and node representing the new model building operation and resulting model, respectively.
\todo[inline]{We may need a special edge so that we know the training operation was not run from scratch and is the result of warmstarting.}

\subsection{Partial Warmstarting Optimization For Model Training Operations}
A common approach in machine learning workloads is to repeatedly select a different subset of features or create new features from the existing ones and train models on the new features.
As a result, many model training operations operate on overlapping or different set of features.
In the partial warmstarting optimization, we aim to improve the training time (and the quality) by warmstarting only the features that exist in the experiment database.

%\subsection{Reuse Optimization for Model Building Operations}
%Reuse for model building operations is more complicated.
%There are two types of reuse opportunities in the model building operations.
%
%\subsubsection{Exact Reuse}\label{sub-sub-exact-reuse}
%For non-user-defined aggregation operations, we follow the same procedure as the feature engineering processes.
%When the corresponding edge in the experiment graph has the same vertex and (aggregation) operation type, we reuse the result of the operation directly.
%We can also reuse the existing model training operation, if the input columns, algorithm, and all the hyper-parameters are the same.
%
%\subsubsection{Model parameter and hyper parameter warmstarting}\label{sub-sub-model-reuse}
%For the model training operations, 3 scenarios can occur.
%In the \textit{first scenario}, the training algorithm used for training the model has never been used before, therefore no meta-data about it exists in the experiment graph.
%In this scenario, no optimization is possible and the model training operation has to be executed completely.
%In the \textit{second scenario}, the training algorithm and the input columns to the model already exist in the experiment graph, but the specific hyperparameter setting does not.
%In this scenario, we can warmstart the model using the parameters from the corresponding node in the experiment graph.
%This reduces the training time as the model \hl{may} converge faster.
%\todo[inline]{This requires experiment and some math ?}
%In the \textit{third scenario}, the training algorithm and the hyperparameters are the same, but all the input columns do not exist in the corresponding node in the experiment graph.
%In this scenario, we provide partial warmstarting.
%In partial warmstarting, the model parameters corresponding to the columns of the input data that already exist in the experiment graph are warmstarted, and the rest of the parameters are randomly initialized.
%\todo[inline]{This requires experiment and some math ?}


%\subsection{Materialization of Grid Search}
%\todo[inline]{Incomplete}
%In order to analyze whether or not we should materialize parts of the grid search, we first have to unpack it, and compare it with other grid search.
%Then, similar to Section \ref{sub-sec-materialization-of-transformed-data}, we materialize the parts that are executed frequently.
%
%%\subsection{Guided Grid-Search}
%%\todo[inline]{just an idea}
%%By extracting correlation between different parameters and the model quality we can provided a guided grid search, where we can provide some estimate or show the effects of a hyperparameter range on the model quality