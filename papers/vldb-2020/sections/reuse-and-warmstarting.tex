\section{Reuse and Warmstarting Optimizations}\label{sec-reuse-and-warmstarting}
With the experiment graph constructed and materialized, we can look for optimization opportunities for feature engineering and model training operations.
In this section, we propose two optimizations, namely, \textit{Reuse}, \textit{Warmstarting}.

\subsection{Reuse Optimization ML Operations}
We devise a strategy to detect overlapping operations in both the current workload and the experiment graph.
If an operation already exists in the experiment graph, we directly access the resulting artifact in the experiment graph instead of executing the operation.
Before executing a workload, we first transform it into its graph representation, which results in a directed graph, called $\mathcal{WG}$.
Then, we traverse the experiment graph starting from the root node, using the edges of the workload graph.
The result of the traversal is a subgraph of the experiment graph, which we refer to as $\mathcal{SG}$.
The subgraph contains all the vertices and edges that exist in both the experiment graph and the workload graph.
For every path in $\mathcal{SG}$ which originates at the root node, we return the furthest materialized vertex from the root node as the result and skip all the intermediate operations.

To demonstrate with an example, let us assume a user has executed the code in Listing \ref{listing-experiment-graph} and Figure \ref{fig-experiment-graph}a shows the current experiment graph, where the set of materialized vertices is $\mathcal{MV} = \{v_0, v_2, v_3\}$.
A new user submits the code in Listing \ref{listing-reuse}.
Figure \ref{fig-reuse}a shows the workload graph ($\mathcal{WG}$) of the code in Listing \ref{listing-reuse}.
As described in the Reuse procedure, we traverse the experiment graph, starting at $v_0$, with the edges of $\mathcal{WG}$ which results in the common subgraph, $\mathcal{SG}$, in Figure \ref{fig-reuse}b.
In the subgraph $\mathcal{SG}$, the furthest vertices in each path originating at $v_0$ are $v_5$ and $v_3$.
The Reuse procedure selects $v_2$ as the candidate since $v_5$ is not materialized.
The Reuse procedure results in an optimized graph that skips operations p2 and p3 are skipped (Figure \ref{fig-reuse}c).

\begin{lstlisting}[language=Python, firstnumber = 8,caption=New workload script (Imports are omitted),captionpos=b,label = {listing-reuse}]
train = pd.read_csv('../input/train.csv') 
selector =  SelectKBest(k=2)
top_features = selector.fit_transform(train[['ts','u_id','price']], 
				      train['y'])
model = svm.SVC()
model.fit(top_features, train['y'])
\end{lstlisting}

\begin{figure}
\captionsetup[subfigure]{justification=centering}
\begin{subfigure}[t]{0.33\linewidth}
\centering
\documentclass{standalone}
\usepackage{tikz}
\usetikzlibrary{graphdrawing, graphs, quotes, positioning,arrows, backgrounds, math, calc}
\usegdlibrary{trees}
\begin{document}
\begin{tikzpicture}
%\draw[help lines]  (-2,0) grid (6,6);
\tikzstyle{every node}=[inner sep=0.02cm]
\node (train) [circle, draw] at (2,6) {$v_0$};
% layer 1
\node (forselection) [circle, draw] at (2,5.2) {$v_2$};
\node (y) [circle, draw] at (3,5.2) {$v_3$};
% layer 2
\node(sk) [circle, draw] at (2, 4.3) {$v_5$};
% layer 3
\node (merged) [circle, draw] at (3, 4.3) {$v_x$};
% layer 4
\node (model) [circle, draw] at (3, 3.4) {$v_y$};


\graph [grow down,edge quotes ={inner sep=1pt}, edges ={thick},radius=.2cm, nodes={circle, draw,font =\small}]{
(train) 
-> [anchor=east,align=center, "p2"] (forselection)
-> [anchor=east,auto=false,align=center,"s1"] (sk)
-> [anchor=north,align=center,"m"](merged) ;

(train) 
-> [anchor=south, align=right,"p3"]   (y)
-> [anchor=west, align=center,"m"](merged) 
-> [anchor=east,auto=false,align=center,"f"] (model);
};
\end{tikzpicture}
\end{document}
\caption{Workload Graph ($\mathcal{WG}$)}
\end{subfigure}%
\begin{subfigure}[t]{0.33\linewidth}
\centering
\begin{tikzpicture}
%\draw[help lines]  (-2,0) grid (6,6);
\tikzstyle{every node}=[inner sep=0.02cm]
\node (train) [circle, draw] at (2,6) {$v_0$};
% layer 1
\node (forselection) [circle, draw] at (2,5.2) {$v_2$};
\node (y) [circle, draw] at (3,5.2) {$v_3$};
% layer 2
\node(sk) [circle, draw] at (2, 4.3) {$v_5$};
% layer 3
\node (merged) [circle] at (3, 4.3) {};
% layer 4
\node (model) [circle] at (3, 3.4) {};


\graph [grow down,edge quotes ={inner sep=1pt}, edges ={thick},radius=.2cm, nodes={circle, draw,font =\small}]{
(train) 
-> [anchor=east,align=center, "p2"] (forselection)
-> [anchor=east,auto=false,align=center,"s1"] (sk);

(train) 
-> [anchor=south, align=right,"p3"]   (y);
};
\end{tikzpicture}
\caption{Common Subgraph ($\mathcal{SG}$)}
\end{subfigure}%
\begin{subfigure}[t]{0.33\linewidth}
\centering
\documentclass{standalone}
\usepackage{tikz}
\usetikzlibrary{graphdrawing, graphs, quotes, positioning,arrows, backgrounds, math, calc}
\usegdlibrary{trees}
\begin{document}
\begin{tikzpicture}
%\draw[help lines]  (-2,0) grid (6,6);
\tikzstyle{every node}=[inner sep=0.02cm]
\node (train) [circle] at (2,6) {};
% layer 1
\node (forselection) [circle, draw] at (2,5.2) {$v_2$};
\node (y) [circle, draw] at (3,5.2) {$v_3$};
% layer 2
\node(sk) [circle, draw] at (2, 4.3) {$v_5$};
% layer 3
\node (merged) [circle, draw] at (3, 4.3) {$v_x$};
% layer 4
\node (model) [circle, draw] at (3, 3.4) {$v_y$};


\graph [grow down,edge quotes ={inner sep=1pt}, edges ={thick},radius=.2cm, nodes={circle, draw,font =\small}]{

 (forselection)
-> [anchor=east,auto=false,align=center,"s1"] (sk)
-> [anchor=north,align=center,"m"](merged) ;

 (y)
-> [anchor=west, align=center,"m"](merged) 
-> [anchor=east,auto=false,align=center,"f"] (model);
};
\end{tikzpicture}
\end{document}
\caption{Optimized Graph}
\end{subfigure}
\begin{subfigure}[t]{\linewidth}
\centering
\begin{tikzpicture}
%\draw[help lines]  (-2,0) grid (6,6);
\tikzstyle{every node}=[inner sep=0.02cm]
\tikzstyle{new} = [fill=blue!25]
\node (train) [circle, draw] at (2,6) {$v_0$};
% layer 1
\node (ad) [circle, draw] at (1,5.2) {$v_1$};
\node (forselection) [circle, draw] at (2,5.2) {$v_2$};
\node (y) [circle, draw] at (4,5.2) {$v_3$};
% layer 2
\node (cv) [circle, draw] at (1, 4.3) {$v_4$};
\node(sk) [circle, draw] at (2, 4.3) {$v_5$};
\node(merged3) [new][circle, draw] at (3, 4.5) {$v_x$};
% layer 3
\node (merged1) [circle, draw] at (1.5, 3.5) {$v_6$};
\node(model2) [new][circle, draw] at (3, 3.7) {$v_y$};
\node (cvsk) [circle, draw] at (2, 2.8) {$v_7$};
% layer 4
\node(merged2) [circle, draw] at (3.7, 2) {$v_8$};
% layer 5
\node(model) [circle, draw] at (3.7,1.1) {$v_9$};

\graph [grow down,edge quotes ={inner sep=1pt}, edges ={thick},radius=.2cm, nodes={circle, draw,font =\small}]{
(train) [label=train]
-> [anchor=east, align=center,"p1"] (ad)
-> [anchor=east,align=center,"v1"] (cv)
-> [anchor=east,align=center,"m"](merged1) 
-> [anchor=east,align=center,"c1"](cvsk) 
-> [anchor=south, align=center,"m"](merged2) ;

(sk) 
-> [anchor=south,align=center,near start,"m"](merged3)
-> [anchor=west,auto=false,align=center,near start, "f"] (model2);

(y)
-> [anchor=south, align=center,"m"](merged3);



(train) 
-> [anchor=east,align=center, "p2"] (forselection)
-> [anchor=east,auto=false,align=center,"s1"] (sk)
-> [anchor=west,align=center,"m"](merged1) ;

(train) 
-> [anchor=west, align=center,"p3"]   (y)
-> [anchor=west, align=center,"m"](merged2) 
-> [anchor=east,auto=false,align=center,"f"] (model);
};
\end{tikzpicture}
\caption{Experiment graph after executing the workload (new nodes are highlighted)}
\end{subfigure}
\caption{Steps in the Reuse optimizations}
\label{fig-reuse}
\end{figure}

\subsection{Warmstarting Optimization For Model Training Operations}
Model training operations include extra hyperparameters that must be set before the training procedure begins.
Two training operations on the same data artifact using the same training algorithm could potentially have very different results based on the values of the hyperparameters.
Therefore, we cannot apply the Reuse optimization in cases the hyperparameters of model training operations are different.
Instead, we apply the \textit{Warmstarting} optimization.
We first need to describe the concept of model groups.
We refer to the set of every machine learning model trained on the same data artifact but only differs in hyperparameters, as a model group.
If a workload contains a model training operation, $e_{m}$, on a vertex, $v_{m}$, before executing the workload, we proceed as follows.
First, using the same traversal strategy as explained in the Reuse procedure, we look for $v_{t}$ in the experiment graph.
If $v_{t}$ is in the experiment graph, then, we find the model group containing all the models  trained on $v_{t}$.
Finally, if the model group is not empty, we warmstart the operation $e_{t}$ with the best performing model from the model group.

To demonstrate with an example, after executing both the code in Listings \ref{listing-experiment-graph} and \ref{listing-reuse}, a new users submits the same code as in Listing \ref{listing-reuse}, with a different model hyperparameters (Line 12) shown in Listing \ref{listing-warmstarting}.
\begin{lstlisting}[language=Python, firstnumber=12, caption= Workload with different hyperparameters,captionpos=b,label = {listing-warmstarting}]
model = svm.SVC(C=0.1)
\end{lstlisting}
Since both models (svm.SVC(C=0.1) and svm.SVC()) are trained on the same vertex ($v_x$ in Figure \ref{fig-reuse}), the Warmstarting procedure selects the existing model node ($v_y$ in Figure \ref{fig-reuse}) to warmstart the training process.

Warmstarting can greatly reduce the total training time.
However, the type of the machine learning model and the termination criteria play important roles in determining the effect of the warmstarting optimization.
In the experiment section, we evaluate the effect of warmstarting on different types of models with different termination criteria.

\subsubsection{Augmenting the experiment graph}
When we utilize warmstarting, we extend the experiment graph with a merge operation which merges the dataset and the candidate model for warmstarting.
The actual training operation is then applied to the merged node.
As a result, we can keep track of the models that are utilized in warmstarting the training of other models which ensures reproducibility.

Figure \ref{fig-warmstarting} shows the experiment graph after execution of the script from Listing \ref{listing-warmstarting}.
The training operation, f1, is applied to the new vertex $v_n$, which is the result of merging the data artifact $v_x$ and the model $v_y$.
The training operation f1 has a different hash from the existing operations since the hyperparameters are different.

\begin{figure}[t]
\centering
\begin{tikzpicture}
%\draw[help lines]  (-2,0) grid (6,6);
\tikzstyle{every node}=[inner sep=0.02cm]
\tikzstyle{new} = [fill=blue!25]
\node (train) [circle, draw] at (2,6) {$v_0$};
% layer 1
\node (ad) [circle, draw] at (1,5.2) {$v_1$};
\node (forselection) [circle, draw] at (2,5.2) {$v_2$};
\node (y) [circle, draw] at (4,5.2) {$v_3$};
% layer 2
\node (cv) [circle, draw] at (1, 4.3) {$v_4$};
\node(sk) [circle, draw] at (2, 4.3) {$v_5$};
\node(merged3) [circle, draw] at (3, 4.5) {$v_x$};
% layer 3
\node (merged1) [circle, draw] at (1.5, 3.5) {$v_6$};
\node(model2) [circle, draw] at (3.5, 3.7) {$v_y$};
\node(merged4) [new][circle, draw] at (2.5, 3.7) {$v_n$};
\node (cvsk) [circle, draw] at (2, 2.8) {$v_7$};
% layer 4
\node(merged2) [circle, draw] at (3.7, 2) {$v_8$};
\node(model3) [new][circle, draw] at (3, 3) {$v_z$};
% layer 5
\node(model) [circle, draw] at (3.7,1.1) {$v_9$};

\graph [grow down,edge quotes ={inner sep=1pt}, edges ={thick},radius=.2cm, nodes={circle, draw,font =\small}]{
(train) [label=train]
-> [anchor=east, align=center,"p1"] (ad)
-> [anchor=east,align=center,"v1"] (cv)
-> [anchor=east,align=center,"m"](merged1) 
-> [anchor=east,align=center,"c1"](cvsk) 
-> [anchor=south, align=center,"m"](merged2) ;

(sk) 
-> [anchor=south,align=center,near start,"m"](merged3)
-> [anchor=west,auto=false,align=center,near start, "f"] (model2);

(y)
-> [anchor=south, align=center,"m"](merged3);

(model2)
->[anchor=north,align=center,"m"](merged4);

(merged3)
->[anchor=west,align=center,"m"](merged4)
-> [anchor=west,align=center,"f1"](model3);

(train) 
-> [anchor=east,align=center, "p2"] (forselection)
-> [anchor=east,auto=false,align=center,"s1"] (sk)
-> [anchor=west,align=center,"m"](merged1) ;

(train) 
-> [anchor=west, align=center,"p3"]   (y)
-> [anchor=west, align=center,"m"](merged2) 
-> [anchor=east,auto=false,align=center,"f"] (model);
};
\end{tikzpicture}
\caption{Experiment graph after warmstarting}
\label{fig-warmstarting}
\end{figure}

%For iterative training algorithms that are minimizing a loss function, there are two termination criteria, namely, the convergence tolerance and the number of iterations.
%
%\subsubsection{Convergence tolerance termination criteria}
%When the termination criteria of the model training operation in the workload is set to a specific convergence tolerance value, two scenarios may occur.
%In the first scenario, an existing trained model in the experiment graph has already reached the convergence tolerance value.
%In this scenario, we expect a large improvement in the training time as the training procedure in the workload will immediately converge.
%In the second scenario, no model in the experiment graph has reached the convergence tolerance value.
%In this case, we warmstart the model in the workload, to the model in the experiment graph with the highest attained quality.
%Therefore, we ensure the training procedure will converge faster.

%\subsubsection{Augmenting the experiment graph}
%Once the training procedure is finished, we augment the experiment graph with an edge and node representing the new model building operation and resulting model, respectively.
%\todo[inline]{We may need a special edge so that we know the training operation was not run from scratch and is the result of warmstarting.}

%\subsubsection{Partial Warmstarting Optimization For Model Training Operations}
%A common approach in machine learning workloads is to repeatedly select a different subset of features or create new features from the existing ones and train models on the new features.
%As a result, many model training operations operate on overlapping or different set of features.
%In the partial warmstarting optimization, we aim to improve the training time (and the quality) by warmstarting only the features that exist in the experiment database.

%\subsection{Reuse Optimization for Model Building Operations}
%Reuse for model building operations is more complicated.
%There are two types of reuse opportunities in the model building operations.
%
%\subsubsection{Exact Reuse}\label{sub-sub-exact-reuse}
%For non-user-defined aggregation operations, we follow the same procedure as the feature engineering processes.
%When the corresponding edge in the experiment graph has the same vertex and (aggregation) operation type, we reuse the result of the operation directly.
%We can also reuse the existing model training operation, if the input columns, algorithm, and all the hyper-parameters are the same.
%
%\subsubsection{Model parameter and hyper parameter warmstarting}\label{sub-sub-model-reuse}
%For the model training operations, 3 scenarios can occur.
%In the \textit{first scenario}, the training algorithm used for training the model has never been used before, therefore no meta-data about it exists in the experiment graph.
%In this scenario, no optimization is possible and the model training operation has to be executed completely.
%In the \textit{second scenario}, the training algorithm and the input columns to the model already exist in the experiment graph, but the specific hyperparameter setting does not.
%In this scenario, we can warmstart the model using the parameters from the corresponding node in the experiment graph.
%This reduces the training time as the model \hl{may} converge faster.
%\todo[inline]{This requires experiment and some math ?}
%In the \textit{third scenario}, the training algorithm and the hyperparameters are the same, but all the input columns do not exist in the corresponding node in the experiment graph.
%In this scenario, we provide partial warmstarting.
%In partial warmstarting, the model parameters corresponding to the columns of the input data that already exist in the experiment graph are warmstarted, and the rest of the parameters are randomly initialized.
%\todo[inline]{This requires experiment and some math ?}


%\subsection{Materialization of Grid Search}
%\todo[inline]{Incomplete}
%In order to analyze whether or not we should materialize parts of the grid search, we first have to unpack it, and compare it with other grid search.
%Then, similar to Section \ref{sub-sec-materialization-of-transformed-data}, we materialize the parts that are executed frequently.
%
%%\subsection{Guided Grid-Search}
%%\todo[inline]{just an idea}
%%By extracting correlation between different parameters and the model quality we can provided a guided grid search, where we can provide some estimate or show the effects of a hyperparameter range on the model quality